\documentclass[11pt]{article}

\usepackage{graphicx} % gere les images 
\usepackage[utf8]{inputenc} % pour les accents
\usepackage{hyperref} % pour les hyperliens
\usepackage[french]{babel} % langue fr
\usepackage{geometry} % pour les marges
\usepackage{listings} % pour le code
\usepackage{color}
\usepackage{enumitem} % permet de changer les listes a puces avec des rond
\usepackage{float}  % pour forcer les images a ce placer ou l'on veux avec l'option H
%\usepackage{here}  % pareil que float

\title{L'affichage de l'enveloppe convexe dans le plan euclidien usuel} 
\author{Armen Mahmedov, Aurelien Patru} 
\date{\today} 

\title{L'affichage de l'enveloppe convexe dans le plan euclidien usuel} 
\author{Armen Mahmedov, Aurelien Patru} 
\date{\today} 


\hypersetup {
	colorlinks=true,
	linkcolor=black,
	urlcolor=red,
	citecolor=green,
	pdftitle={Initiation aux outils de simulation réseaux avec NS-2},
	pdfauthor={Armen,Aurelien},
	pdfsubject={simulation réseau avec NS-2},
	pdfkeywords={NS-2, réseau, nam, esirem, tcp, udp, }
}


% ################### Configuration partie code ########################### %
\definecolor{darkWhite}{rgb}{0.94,0.94,0.94}
\lstset{
	aboveskip=3mm,
	belowskip=-2mm,
	%backgroundcolor=\color{darkWhite},
	basicstyle=\footnotesize\ttfamily,
	breakatwhitespace=false,
	breaklines=true,
	captionpos=b,
	deletekeywords={...},
	escapeinside={\%*}{*)},
	extendedchars=true,
	keepspaces=true,
	%keywordstyle=\color{blue},
	literate=
	{²}{{\textsuperscript{2}}}1
	{⁴}{{\textsuperscript{4}}}1
	{⁶}{{\textsuperscript{6}}}1
	{⁸}{{\textsuperscript{8}}}1
	{€}{{\euro{}}}1
	{é}{{\'e}}1
	{è}{{\`{e}}}1
	{ê}{{\^{e}}}1
	{ë}{{\¨{e}}}1
	{É}{{\'{E}}}1
	{Ê}{{\^{E}}}1
	{û}{{\^{u}}}1
	{ù}{{\`{u}}}1
	{â}{{\^{a}}}1
	{à}{{\`{a}}}1
	{á}{{\'{a}}}1
	{ã}{{\~{a}}}1
	{Á}{{\'{A}}}1
	{Â}{{\^{A}}}1
	{Ã}{{\~{A}}}1
	{ç}{{\c{c}}}1
	{Ç}{{\c{C}}}1
	{õ}{{\~{o}}}1
	{ó}{{\'{o}}}1
	{ô}{{\^{o}}}1
	{Õ}{{\~{O}}}1
	{Ó}{{\'{O}}}1
	{Ô}{{\^{O}}}1
	{î}{{\^{i}}}1
	{Î}{{\^{I}}}1
	{í}{{\'{i}}}1
	{Í}{{\~{Í}}}1,
	morekeywords={*,...},
	framexleftmargin=16pt,
	framextopmargin=3pt,
	framexbottommargin=6pt,
	frame=td,
	numbers=left,    % numerotation
	numbersep=10pt,
	numberstyle=\tiny\color{black},
	stepnumber=1,
	rulecolor=\color{black},
	showspaces=false,
	showstringspaces=false, % don't mark spaces in strings
	showtabs=false,
	%stringstyle=\color{red},
	title=\lstname,
	keywordstyle=\color{red}\bfseries, 
 	commentstyle=\color{blue}\textit,
 	stringstyle=\color{cyan}\ttfamily,
 	tabsize=2,
 	frame=single % or td
}

% Pour changer les listings en Code
\renewcommand\lstlistingname{Script}
\renewcommand\lstlistlistingname{Script}

\setlist[itemize]{label=\textbullet} % pour tous les itemize changer le tiret avec un rond


\begin{document}

\begin{titlepage}
	\newcommand{\HRule}{\rule{\linewidth}{0.2mm}}     
            
	\begin{figure}[t]
		\begin{minipage}{0.5\textwidth}\large
			\begin{flushleft}
				\includegraphics[width=0.7\textwidth]{assets/logoEsirem.jpg}
			\end{flushleft}
		\end{minipage}
		\begin{minipage}{0.5\textwidth}\large
			\begin{flushright}
			\includegraphics[width=0.6\textwidth]{assets/logoUb.jpg}
			\end{flushright}
		\end{minipage}
	\end{figure}
	\textsc{ \\[1cm]}
     
     
	\begin{center}
	\HRule \\
	{\Large   
		Initiation aux outils de simulation réseaux avec NS-2
	}
	\HRule
	\\[0.5cm]
	{\large Compte rendu tp reseau \\}
   \end{center}

	\begin{figure}[h]
		\begin{center}
			\includegraphics[width=0.5\textwidth]{assets/main.jpg}
		\end{center}
	\end{figure}
        
	\begin{center}
        \textsc{\large
        	ESIREM \\
			École Supérieure d'Ingénieurs de Recherche en Matériaux et en Infotronique\\       	
			[1.5cm]
		}
	\end{center}

	\begin{minipage}{0.5\textwidth}
		\begin{flushleft} \large
			\emph{Auteurs:}\\
			Armen Mahmedov \\
			Aurélien Patru
		\end{flushleft}
	\end{minipage}
	~
	\begin{minipage}{0.5\textwidth}
		\begin{flushright}\large
			\emph{Professeurs :} \\
			Nader Mbarek \\
			Ahmad Khalil
		\end{flushright}
	\end{minipage}
        
\end{titlepage}
  
% saut de page  ou   ~ \pagebreak 
\newpage
\strut
\newpage

%------------------------
%     Table des matieres
%------------------------
\tableofcontents

\pagebreak

%------------------------
%    Table des figures
%------------------------
\listoffigures

\pagebreak


\section*{Introduction}
\addcontentsline{toc}{section}{Introduction}
Introduction générale 

Le but des tps reseaux

Presenter ns2

[A quoi sert la simulation des reseaux? Avantage]

- permet de simuler un upgrade de la structure actuelle sans toucher ou staturer le reseau.   

- permet de faire des simulations avant d'acheter un nouveau matérielle 


\pagebreak


% Partie 1
\section{TP 1: Découvert de NS-2 et nam}
[introduction de la partie] **** dire le but du tp  *****

\subsection{Exercice 1: Simulation d'une topologie simple a deux noeuds avec un lien direct}

Dans un premier temps le but est de compléter le script fournis pour comprend se familiariser avec le language et comprendre son fonctionnent

\subsubsection{Compléter le script}
Pour créer deux nœuds on utilise la commande \textbf{set}. Pour cela on peut soit créer les nœuds un par un soit le faire a l'aide d'un boucle for :

\begin{lstlisting}[language=tcl, numbers=none, framexleftmargin=0pt, framextopmargin=0pt, framexbottommargin=0pt]
set NodeNb 2
for {set i 0} {$i<$NodeNb} {set i [expr $i+1]} {set n($i) [$ns node]}
\end{lstlisting}

\noindent % pour ne pas faire d'alinea
Pour relier les deux nœuds on utilisera les paramètres suivants:
\begin{itemize}
	\item lien duplex de capacité 1MB
	\item temps de propagation de 10ms
	\item file d'attente de type DropTail
\end{itemize}

\begin{lstlisting}[language=tcl, numbers=none, framexleftmargin=0pt, 	framextopmargin=0pt, framexbottommargin=0pt]
$ns duplex-link $n(0) $n(1) 1Mb 10ms DropTail
\end{lstlisting}


Créer un agent UDP et l'attacher a n0. Ici la variable udp est une instance de la classe UDP qui hérite de la classe Agent

\begin{lstlisting}[language=tcl, numbers=none, framexleftmargin=0pt, 	framextopmargin=0pt, framexbottommargin=0pt]
set udp [new Agent/UDP]
#atacher l'aplication udp au noeud 0
$ns attach-agent $n(0) $udp 
\end{lstlisting}


Ensuite il faut créer une source de trafic CBR (\textit{constant bit rate}) avec les paramètres suivant:
\begin{itemize}
	\item taille des paquets de 500 octets
	\item intervalle de transmission de paquets de 0.005s
\end{itemize}

\begin{lstlisting}[language=tcl, numbers=none, framexleftmargin=0pt, 	framextopmargin=0pt, framexbottommargin=0pt]
#creattion d'un source de trafique
set cbr [new Application/Traffic/CBR]
$cbr set packetSize_ 500bytes
$cbr set interval_ 0.005s
\end{lstlisting}


Après avoir créer la source CBR, il faut maintenant le connecter a l'agent UDP avec la commande \textbf{attach-agent}
\begin{lstlisting}[language=tcl, numbers=none, framexleftmargin=0pt, 	framextopmargin=0pt, framexbottommargin=0pt]
#connecter le cbr a l'agent UDP
$cbr attach-agent $udp
\end{lstlisting}


Ensuite on doit créer l'agent null pour la réception des paquets dans le noeud 1 et connecter les deux agents Null et UDP
\begin{lstlisting}[language=tcl, numbers=none, framexleftmargin=0pt, 	framextopmargin=0pt, framexbottommargin=0pt]
#creation de l'agent null
set agentNull [new Agent/Null]
#atacher l'agent null au noeud 1
$ns attach-agent $n(1) $agentNull 
#connecter l'agent null et l'agent udp
$ns connect $udp $agentNull 
\end{lstlisting}

Il ne reste plus qu'à déclencher le trafic CBR à t=1s et l’arrêter à t=4.5s. Pour finir on doit arrêter la simulation après 5s. 
\begin{lstlisting}[language=tcl, numbers=none, framexleftmargin=0pt, 	framextopmargin=0pt, framexbottommargin=0pt]
#declancher le trafic cbf
$ns at 1 "$cbr start"
$ns at 4.5 "$cbr stop"
#Appeler la procédure de terminaison apres un temmps t 
$ns at 5.0 "finish"
#Executer la simulation
$ns run
\end{lstlisting}



\noindent
Le script complet se trouve l'annexe \ref{tp1Exo1}

\subsubsection{Visualisation de la simulation a l'aide de NAM}
Une fois le script complété on ouvre un terminal et on tape la commande \textbf{ns exercice1.tcl} pour compiler et lancer l'interface graphique nam que l'on peut voir dans la figure \ref{namVisu}. Cette commande crée aussi un fichier out.nam qui a été définit plus haut dans le script, plus tard si on veux revoir la simulation avec nam il suffira de lancer la commande \textbf{nam out.nam}


\subsubsection{Découvert des fonctionnalités de NAM}
L'utilitaire s'ouvre sur l'interface présentée en Figure \ref{namVisu}. Pour mieux comprendre le fonctionnement de nam on va détailler les différents parties présent en rouge sur la figure.

\begin{figure}[H]
	\begin{center}
		\includegraphics[width=0.7\textwidth]{assets/tp1/nam-visualisation1.png}
	\end{center}
	\caption{Interface nam pour la visualisation de la simulation}
	\label{namVisu}
\end{figure}

\noindent
(1) Sert a contrôler le déroulement la simulation \\
(2) Sert a visualiser le temps de la simulation et permet de contrôler sa vitesse\\
(3) Interface graphique ou la simulation se passe \\
(4) Permet de visualiser l'avancement de la simulation par rapport a la duré de la simulation et permet de se rendre directement un moment de la simulation\\
(5) Permet de changer la disposition des nœuds dans le cas ou les liens se croisent ou de reset la disposition des liens\\


********** Mettre les images présent dans le dossier TP1 **************************

\subsubsection{Utilisation graphique de NAM}

Maintenant il faut simuler le même réseau que dans la partie 1.1.1 mais cette fois avec l’interface graphique de nam. Pour cela nous allons tout d'abord faire une présentation rapide de l'interface graphique qui permet de créer la simulation. Pour lancer le logiciel il suffit de taper nam dans le terminal, ensuite il suffit d'aller dans fichier et cliquer sur nouveau éditeur nam,  l'utilitaire s'ouvre sur l'interface présentée en Figure \ref{namExplcation}.

\begin{figure}[H]
	\begin{center}
		\includegraphics[width=0.8\textwidth]{assets/tp1/explicationNam.png}
	\end{center}
	\caption{L'interface graphique nam}
	\label{namExplcation}
\end{figure}

\noindent
(1) Le menu permettant d’accéder aux différents fonctions proposé par le logiciel\\
(2) Le menu d'action rapide qui permet d'avoir un raccourcis vers les fonctions présents dans le menu Tools\\
(3) L'interface graphique ou il faut placer les différents objets \\
(4) Le  curseur qui permet de sélectionner les objets et les bouger\\
(5) L'option qui permet de créer des nœuds\\
(6) Permet de faire le liens entre 2 nœuds ou 2 agents\\
(7) Permet de choisir un agent\\
(8) Permet de choisir une source de trafic \\
(9) Zoom/Dézoom\\
(10) Exemple de nœuds avec un agent TCP et une source de trafic FTP connecté a FTP\\
(11) Exemple de nœuds avec un agent TCPSink\\

Dans un premier temps on va créer 2 nœuds a l'aide du bouton ref(5) figure \ref{namExplcation}. Ensuite on les relis a l'aide du bouton ref(6) et avec in clic droit sur le lien on peut choisir le débit (Bandwidth), le temps de propagation (Delay) et la couleur du lien comme on peut le voir dans la figure \ref{ajoutLien}. Pour le type de file d'attente on peut le définir avec un clic droit le sur Q présent dans la figure \ref{ajoutLien}

\begin{figure}[H]
	\begin{center}
		\includegraphics[width=0.6\textwidth]{assets/tp1/ajoutLienNoeud.png}
	\end{center}
	\caption{Ajout de lien entre 2 nœuds}
	\label{ajoutLien}
\end{figure}

Ensuite on crée l'agent UDP en sélectionnant dans le liste présent dans le ref(7) figure \ref{namExplcation} l'agent UDP et on cliquant sur le nœud 1. Pour pouvoir envoyer des informations il faut ajouter une source de trafic, ici on va ajouter une source CBR a notre agent UDP a l'aide du ref(8) figure \ref{namExplcation}, avec un clic droit sur la source on peut modifier l'intervalle de trasmission des paquets et quand le trafic doit se déclencher et s’arrêter comme on peut le voir dans la figure \ref{modifCBR}. Ensuite on va jouter l'agent Null au nœud 2 et connecter l'agent null avec l'agent UDP, après cette opération on peut modifier les options de UDP et changer la taille des paquets a 500 octets avec un clic droit sur l'agent UDP. Enfin pour modifier le temps de fin de simulation, il faut aller dans le menu \textit{Éditer} puis dans \textit{Propriétés du Simulateur}.

\begin{figure}[H]
	\begin{center}
		\includegraphics[width=0.6\textwidth]{assets/tp1/modifCBR.png}
	\end{center}
	\caption{Configuration source CBR}
	\label{modifCBR}
\end{figure}

La création de simulation avec l'interface graphique comporte quelques défauts comme par exemple la taille des paquets (packetSize) qui ne peut pas dépasser 210 octets alors qu'on les a fixer a 500.

\subsection{Exercice 2: Fonctionnement TCP vs UDP}

Dans cette exercice on s’intéresse au fonctionnement de TCP et UDP, pour cela on va simuler le réseau de la figure \ref{resSimu}.

\begin{figure}[H]
	\begin{center}
		\includegraphics[width=0.6\textwidth]{assets/tp1/reseauAFaire.png}
	\end{center}
	\caption{Réseau à réaliser}
	\label{resSimu}
\end{figure}

Le script complet de cette exercice est disponible dans l'annexe \ref{tp1Exo2}. Par la suite nous ne détaillerons que les nouveaux éléments qui n'ont pas encore été abordé.

Pour réaliser ce réseau, on va tout d'abord créer 4 nœuds avec un boucle for, ensuite on les relis avec un lien duplex de capacité 2Mb avec un temps de propagation de 10ms pour les liens entre les nœuds n0 - n2 et n1 -n2 et un lien de 1,7Mb et un temps de propagation de 20ms entre n2 - n3, avec une file d'attente de type DropTail.
Par la suite on va créer 4 agent: 
\begin{itemize}
	\item TCP
	\item UDP
	\item Null
	\item TCPSink
\end{itemize}
L'agent TCPSink servira a communiquer avec le protocole TCP et l'agent Null avec le protocole UDP. De plus on va attacher l'agent TCP au nœud n0, UDP au nœud n1, TCPSink et Null au nœud n3. Ensuite on va créer 2 applications qui serviront a envoyer des données. L'application FTP qui sera connecté a l'agent TCP et le CBR à UDP.\\

Pour la partie configuration, on va limiter la taille du buffer entre n2 et n3 pour avoir une limité de 10 et pouvoir visualiser une congestion  avec la commande 
\begin{lstlisting}[language=tcl, numbers=none, framexleftmargin=0pt, 	framextopmargin=0pt, framexbottommargin=0pt]
$ns queue-limit $n(2) $n(3) 10
\end{lstlisting}

Pour l'agent CBR on va changer la taille des paquets a 1000 octets et un débit de 1Mb/s
\begin{lstlisting}[language=tcl, numbers=none, framexleftmargin=0pt, 	framextopmargin=0pt, framexbottommargin=0pt]
$cbr set packetSize_ 1000bytes
$cbr set rate_ 1Mb
\end{lstlisting}

Pour positionner les nœuds on va utiliser la commande \textbf{duplex-link-op}.
\begin{lstlisting}[language=tcl, numbers=none, framexleftmargin=0pt, 	framextopmargin=0pt, framexbottommargin=0pt]
#positionner les noeuds
$ns duplex-link-op $n(0) $n(2) orient right-down
$ns duplex-link-op $n(1) $n(2) orient right-up
$ns duplex-link-op $n(2) $n(3) orient right
\end{lstlisting}

Pour mieux visualiser et pouvoir distinguer les segments TCP et les datagrammes UDP on va les colorier, TCP en rouge et UDP en bleu.
\begin{lstlisting}[language=tcl, numbers=none, framexleftmargin=0pt, 	framextopmargin=0pt, framexbottommargin=0pt]
#déclaration de la couleur
$ns color 1 Red
$ns color 2 Blue

#affecter couleurs
$tcp set class_ 1
$udp set class_ 2
\end{lstlisting}



Pour finir on va lancer le trafic CBR à 0.1s et le terminer a 4.5s, le trafic TFP se lancera a 1s et se terminera a 4s

\begin{lstlisting}[language=tcl, numbers=none, framexleftmargin=0pt, 	framextopmargin=0pt, framexbottommargin=0pt]
#declancher le trafic cbf
$ns at 0.1 "$cbr start"
$ns at 4.5 "$cbr stop"
#declancher le trafic ftp
$ns at 1 "$ftp start"
$ns at 4 "$ftp stop"
\end{lstlisting}


Après la complication on se trouve dans nam qui compote notre simulation et on peut distinguer différentes étapes lors de son exécution.

\begin{enumerate}
	\item Envoie directe des données par UDP a 0.1s figure \ref{EnvoieUDP}
	
\begin{figure}[H]
	\begin{center}
		\includegraphics[width=0.6\textwidth]{assets/tp1/envoieDonneUDP.png}
	\end{center}
	\caption{Envoie de donné par UDP}
	\label{EnvoieUDP}
\end{figure}

	\item Envoie d'un demande de connexion par TCP à 1s (n0 demande à n3) figure \ref{demandeCo}. Le nœud n3 accepte la connexion figure \ref{acceptCo}
	
\begin{figure}[H]
	\begin{center}
		\includegraphics[width=0.6\textwidth]{assets/tp1/demandeDeCo.png}
	\end{center}
	\caption{Demande de connexion TCP}
	\label{demandeCo}
\end{figure}

\begin{figure}[H]
	\begin{center}
		\includegraphics[width=0.6\textwidth]{assets/tp1/accepteCo.png}
	\end{center}
	\caption{Accepte demande de connexion TCP}
	\label{acceptCo}
\end{figure}

	\item Envoie des premiers segments TCP après l’établissement de la connexion. Ensuite envoie des segments seulement après acquittement figure \ref{acuseRecep}

\begin{figure}[H]
	\begin{center}
		\includegraphics[width=0.6\textwidth]{assets/tp1/acuseReceptionPaquet.png}
	\end{center}
	\caption{Accusé de réception d'un paquet}
	\label{acuseRecep}
\end{figure}
	
	\item Congestion au niveau du nœud 2 avec drop de segment
	
\begin{figure}[H]
	\begin{center}
		\includegraphics[width=0.6\textwidth]{assets/tp1/dropDePaquet.png}
	\end{center}
	\caption{Perte de paquet}
	\label{acuseRecep}
\end{figure}
	
	\item Après la perte des segments, UDP continu d'envoyer comme avant alors que TCP ayant détecté la perte 	réagit et diminue la quantité d’information envoyé, figure \ref{detectPaquete}

\begin{figure}[H]
	\begin{center}
		\includegraphics[width=0.6\textwidth]{assets/tp1/tcpDetectionPerte.png}
	\end{center}
	\caption{Détection de perte de la part de TCP}
	\label{detectPaquete}
\end{figure}
	
	
	
\end{enumerate}


Au départ TCP envoie que 2 segments, après l’acquittement de ces segments, TCP va doubler la taille de la fenêtre et envoyer 4 et ainsi de suite jusqu'au moment ou il détecte une perte de segment et va réagir en diminuant la quantité d'information envoyé pour s'adapter au capacité du réseau (contrôle de congestion) figure \ref{detectPaquete}. 

Au bout d'un moment nœud 2 reçoit trop d'information en même temps des nœuds n0 et n1 (4Mb/s) alors qu'il transmet qu'a 1.7MB/s, il est donc obligé de stocker les segments dans son buffer qui a une capacité limité (10). Quand le buffer du nœud n2 atteint sa capacité maximale le nœud commence a doper les paquets, on peut apercevoir ce comportement dans la figure 

~\\
\\
\\
\\
\\
\\
\\
\\
\\


Pour conclure dans ce premier TP, nous avons pu nous familiariser avec l’outil de simulation NS-2. Avec tout d’abord la réalisation d'un réseau à l’aide d’un script OTCL permettant de découvrir le langauge et son fonctionnement, puis la visualisation la création de ce même réseau avec l'éditeur graphique nam permettant ainsi de se familiariser avec cette outil. Dans un deuxième temps nous avons pu comparer deux modes de communications (TCP et UDP) vu en cours pour mieux comprendre leur fonctionnement. Nous avons pu identifier le comportement de ces 2 modes de communication en cas de congestion ou encore visualiser l'envoie de données. Pour le protocole TCP nous avons visualisé l'envoie des segment TCP, la demande de connexion et l'acquittement. Pour le protocole UDP nous avons vu le mode non connecté et l'envoie de segment UDP (datagrame).

\pagebreak
% Partie 2
\section{TP 2: Protocole de routage de type Vecteur distance et Etat de liens}
[introduction de la partie]

\subsection{Exercice 1: Comparaison des deux protocoles de routage}


\subsection{Exercice 2: Routage et contrôle de congestion TCP}

[conclusion de la partie]





\pagebreak
% Partie 3
\section{TP 3: Utilisation de xgraph et NS-2 pour visualiser les performances d'un réseau}
[introduction de la partie]

\subsection{Sous-partie 1}

[conclusion de la partie]




\pagebreak
\section*{Conclusion}
\addcontentsline{toc}{section}{Conclusion}
[Conclusion générale]


\pagebreak
\section*{Annexes}
\addcontentsline{toc}{section}{Annexes}

\begin{lstlisting}[language=tcl, label={tp1Exo1}, caption={TP1 - Exercice1}]
#Création d'une instance de l'objet Simulator
set ns [new Simulator]

#Ouvrir le fichier trace pour nam
set nf [open out.nam w]
$ns namtrace-all $nf

#Définir la procedure de terminaison de la simuatation
proc finish {} {
	global ns nf
	$ns flush-trace
	#fermer le fichier trace
	close $nf
	#Exectuer le nam avec en entrée le fichier trace
	exec nam out.nam &
	exit 0
}

#Insérer votre code
set NodeNb 2
for {set i 0} {$i<$NodeNb} {set i [expr $i+1]} {set n($i) [$ns node]}
#connecter les deux noeuds
$ns duplex-link $n(0) $n(1) 1Mb 10ms DropTail
set udp [new Agent/UDP]
#atacher l'aplication udp au noeud 0
$ns attach-agent $n(0) $udp 
#creattion d'un source de trafique
set cbr [new Application/Traffic/CBR]
$cbr set packetSize_ 500bytes
$cbr set interval_ 0.005s
#connecter le cbr a l'agent UDP
$cbr attach-agent $udp

#creation de l'agent null
set agentNull [new Agent/Null]
#atacher l'agent null au noeud 1
$ns attach-agent $n(1) $agentNull 
#connecter l'agent null et l'agent udp
$ns connect $udp $agentNull 
#declancher le trafic cbf
$ns at 1 "$cbr start"
$ns at 4.5 "$cbr stop"

#Appeler la procédure de terminaison apres un temmps t 
$ns at 5.0 "finish"

#Executer la simulation
$ns run
\end{lstlisting} 

~\\

\begin{lstlisting}[language=tcl, label={tp1Exo2}, caption={TP1 - Exercice2}]
#Création d'une instance de l'objet Simulator
set ns [new Simulator]

#Ouvrir le fichier trace pour nam
set nf [open out2.nam w]
$ns namtrace-all $nf

#Définir la procedure de terminaison de la simuatation
proc finish {} {
	global ns nf
	$ns flush-trace
	#fermer le fichier trace
	close $nf
	#Exectuer le nam avec en entrée le fichier trace
	exec nam out2.nam &
	exit 0
}

#Insérer votre code
set NodeNb 4
for {set i 0} {$i<$NodeNb} {set i [expr $i+1]} {set n($i) [$ns node]}

#connecter les deux noeuds
$ns duplex-link $n(0) $n(2) 2Mb 10ms DropTail
$ns duplex-link $n(1) $n(2) 2Mb 10ms DropTail
$ns duplex-link $n(2) $n(3) 1.7Mb 20ms DropTail

#creation des agents
set tcp [new Agent/TCP]
set udp [new Agent/UDP]
set agentNull [new Agent/Null]
set tcpSink [new Agent/TCPSink] 

#creation des applications
set ftp [new Application/FTP]
set cbr [new Application/Traffic/CBR]

#attacher agent aux noeuds
$ns attach-agent $n(0) $tcp
$ns attach-agent $n(1) $udp
$ns attach-agent $n(3) $tcpSink
$ns attach-agent $n(3) $agentNull

#connecter les applications aux agents 
$cbr attach-agent $udp
$ftp attach-agent $tcp

#connecter les agents avec les agents coresspondant
$ns connect $udp $agentNull 
$ns connect $tcp $tcpSink 

#configurations 
$ns queue-limit $n(2) $n(3) 10

$cbr set packetSize_ 1000bytes
$cbr set rate_ 1Mb

$ns duplex-link-op $n(0) $n(2) orient right-down
$ns duplex-link-op $n(1) $n(2) orient right-up
$ns duplex-link-op $n(2) $n(3) orient right

#coloration
$ns color 1 Red
$ns color 2 Blue
$ns color 3 Green

$tcp set class_ 1
$udp set class_ 2
#$tcpSink set class_ 3

#declancher le trafic cbf
$ns at 0.1 "$cbr start"
$ns at 4.5 "$cbr stop"
#declancher le trafic ftp
$ns at 1 "$ftp start"
$ns at 4 "$ftp stop"

#Appeler la procédure de terminaison apres un temmps t 
$ns at 5.0 "finish"

#Executer la simulation
$ns run
\end{lstlisting}

\end{document}
